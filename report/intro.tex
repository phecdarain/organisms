\section{Introduction}

This report describes our approach and observations while working on the 
{\em Organisms} simulation problem. The simulation is a class of popular
{\em Discrete Cellular Automaton} model, similar in spirit to 
{\em Conway's Game of Life}.

The simulation world is a finite grid in which virtual organisms occupy 
individual cells in the grid. These virtual organisms can choose to move, 
eat or reproduce in each discrete cycle of the simulation. They are also 
able to "see" their neighboring cells and are thus able to sense the presence 
of food and/or other organisms. The simulation is subject to various rules 
and  constraints. Each organism has finite energy and there is a cap on 
maximum energy a organism can accumulate. Energy is replenished by consuming 
food and it is consumed in every cycle depending on organism's action - 
i.e. moving, reproducing or even staying still consumes some finite amount 
of energy.  An organism ceases to exist when it's energy level drops to zero.
Food may appear on any empty cell in the simulation world with some finite
probablility and it may increase by a unit with some finite probability. 
These probabilities are not known to the organism. The maximum amount of 
food that is available in any cell is also bound to a finite number. 
An organism may also "communicate" with nearby organisms by broadcasting an 
external state, that can used in a versatile manner.
The simulation can be run in single-species mode or multiple-species mode. 
The winning conditions are not concretely defined but some of the factors 
that can be considered while judging a winning strategy can be 
average energy per organism, average extinction rate, total number of
organisms for a species vis-a-vis other species etc.
