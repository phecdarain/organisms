\section{Introduction}

This report describes our approach and observations while working on the 
{\em Organisms} simulation problem. The simulation is a class of popular
{\em Discrete Cellular Automaton} model, similar in spirit to 
{\em Conway's Game of Life}.

The simulation world is a finite grid in which virtual organisms occupy 
individual cells in the grid. These virtual organisms can choose to move, 
eat or reproduce in each discrete cycle of the simulation. They are also 
able to "see" the neighboring cells and are thus able to sense the presence 
of food and/or other organisms. The simulation is subject to various rules 
and  constraints. Each organism has finite energy and there is a cap on 
maximum energy an organism can accumulate. Energy is replenished by consuming 
food and it is consumed in every cycle depending on organism's action - 
i.e. moving, reproducing or even staying still consumes some finite amount 
of energy.  An organism ceases to exist when it's energy level drops to zero.
Food may appear on any empty cell in the simulation world with some finite
probability and it may increase by a unit with some finite probability. 
These probabilities are not known to the organism. The maximum amount of 
food that is available in any cell is also bound to a finite number. 
An organism may also "communicate" with nearby organisms by broadcasting an 
external state, that can used in a versatile manner.
The simulation can be run in single-species mode or in multiple-species mode. 
The winning conditions are not concretely defined but some of the factors 
that can be considered while judging a winning strategy can be 
average energy per organism, average extinction rate, total number of
organisms for a species vis-a-vis other species etc. after some finite 
number of rounds/cycles.

Our group came up with various approaches to the simulation and depending on 
results, we pivoted a few times. Initially, our approach focused primarily 
to maximize the average energy. We developed an organism brain which used
different {\em threshold} levels to decide whether to move, reproduce or to 
stay still. The implementation assumed fairly high probability
of food appearing on the board. However, the strategy fell flat in conditions 
where food was scarce. We then implemented a strategy in which the organisms 
tried to gauge the probability of food appearing and doubling, either 
individually or collectively using some means of communication. We also came 
up with mathematical equations to determine the "utility" of making a move, 
consuming food, moving or staying still in each cycle of the simulation. 
The utility functions allowed us to choose the "best" move for an organism. 

However, the release of {\em "Sheep"} player from the previous 
undertaking of this class completely changed our thinking of this game. 
We realized that our strategies were not agile enough to adapt to 
multi-species simulations and our organism species were being constantly 
starved into extinction by the fairly aggressive reference player. 
We discarded the old approach in favor of the newer {\em "Flood"} based 
approach, which aimed to reproduce quickly in the simulation arena so as 
to avoid possible extinction by other players. 

