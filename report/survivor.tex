\subsubsection{Surviving Mode}
One thing we noticed is that for harsh environment(p = 0.1\%), the
organism will easily extinct if moving around too much, especially at
the beginning of game.  However, if we are able to let the food to
grow for a while, the board will become abundant, which then allows
the organism to make some move and maximize total power.  Basically,
we attempt to help our organism \textit{Survive} the beginning couple
hundred rounds(Stage One) and switch to energy miximizing mode(Stage
Two). Note that the surviving mode is just one layer put at the
beginning of the game. At some point(details in Implementation), when
organism enters stage two, \textit{any} algorithm for maximizing energy could
be invoked. The two stages are totally decoupled.

\textbf{Move or stay.}  During this surviving mode, apparently,
reproduce is not a good action.  The energy spliting makes the
origanism weak and cost of stayput will doubled.  The question
remained to be whether the organism should just stay, waiting for food
to appear in the neighbor cells or move around to explore food. We
answer this question by introducing a probabilistic analysis over the
benefit of moving or stay.  More specifically, we will calculate the
probability of finding food when you move and compare it with the
probablity of finding food if those energy are used to stayput and
wait, so as to determine the best action in each round.

\textbf{Eat or wait.} The idea above is focusing on finding one food.
But once a food is found in neighbor cell, shall the organism just go
ahead and eat it or wait for as long as it can. Our answer is the
later, based the following three reasons.
\begin{enumerate}
\item This is single player mode, no one will appear and rob the food
  away.
\item The goal of this stage is to survive the beginning harsh
  environment, waiting is more suitable to this.
\item The expectation of food on that cell grows exponential to the
  time you wait(more details in Implementation).
\end{enumerate}

