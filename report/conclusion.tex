\section{Conclusion}
\label{sec:conclusion}
Our ultimate findings can be summarized in two statements:

\begin{enumerate}
\item Multiplayer strategies tend to perform better when they are initially-aggressive ultimately-conservative behavior-based organisms, and
\item Single-player boards are best played using a structural approach which measures and adjusts according to board variables.
\end{enumerate}

We found it difficult to combine the two strategies into an organism
which detects board conditions and responds accordingly; however, we
do believe that such a strategy is ultimately possible.  Nonetheless,
it seems that the optimal single-player and multiplayer strategies are
orthogonal to each other: without sophisticated detection and
communication algorithms, an organism must choose to specialize to
either a singleplayer or multiplayer role.

While we were not successful in combining the strategies, we were
generally pleased with the success of our individual strategies.  The
single-player oriented PatternMaker achieved 95\% of the maximal
potential board energy; meanwhile, the multi-player oriented Flood was
able to outperform other organisms in its designed setup(win
overwhelmingly on $50\times 50$ board). Moreover, it was able to
frequently compete and succeed against multiple copies of the
SurviveSheep organism simultaneously.

Throughout the project, we found that our strategies did change
considerably, greatly aided by new perspectives gained through class
discussions.  We found that frequently reconsidering our assumptions
and approaching the problem from totally new perspectives was
enormously helpful in moving forward.  Contrary to class sentiment, we
felt that overarching strategy, not parameters, still accounted for
the primary differences between organisms.
