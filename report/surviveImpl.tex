\subsection{Surviving Mode}
Recall that the goal of this surviving mode is to live through 
the beginning stage of the game, when there are very little food on the ground
and switch to the mode that maximize energy(pattern maker).

\textbf{Move or stay.} To determine whether it is better to stay where we are or 
to move, we try to calculate the payoff of each actions. 
More specifically, we will calculate the probability of \textit{not} finding 
any food if (a) spend $v$ energy to move or (b) say the energy to stay for $\frac{v}{s}$.
Assume the organism has already wait for $t$ time and there is still 
no food observed in the neighbor cells. 
\begin{itemize}
\item If moving, in the next round, the organism will be able to see three new cells each having probability of $(1-p)^{t+1}$ of no food grown.
Remember that the organism leave the original cell at time(round) $t$, 
which allows food to grow on this cell at $t+1$ round.
Therefore, overall, the probability of finding no food after move is $(1-p)^{3(t+1)+1}$,
i.e., probability of no food observed on the three new cells and no food grows on the original cell.

\item If stay for $\frac{v}{s}$ rounds, each round, the probability of no food grows in each of the
four neighbor cells is $(1-p)$. 
Thus the overall probability of no food observed after those rounds is $(1-p)^{\frac{4v}{s}}$.
\end{itemize}
According to the above analysis, apparently, the organism should stay at the beginning 
until the time $3(t+1)+1 > \frac{4v}{s}$($(1-p)$ is smaller than $1$, thus larger index yields 
smaller probability). 
Note that an interesting property of this formula is that the analysis depends on $p$ 
while the final formula itself does not, which is important in this case since $p$ is not given.
Once the organism starts moving, it won't stop and will keep following one direction, simply 
because make any turn will lose the chance to see an entirely new cell.

\textbf{Eat or wait.} Once the organism find a food nearby, it will wait for as long as it can, 
and then eat the food. The three reasons of this has already been discussed in previous section.
Now we briefly explain why the expected amount of food is exponential to the time you wait(more specifically, $(1+q)^t$).
Note that this is not concrete proof, just reasoning by intuition.
First of all, the simplest case, when there is one food, in the next round, the expectation is $q \times 2 + (1-q) \times 1 = 1 + q$.
Using induction, at time $t$, assume the expectation of food is $(1+q)^t$. 
The expectation could be written as $\sum_i p_i \times i$ where $p_i$ is the probability of there being $i$ food on cell. For $i$ food, each of them has expectation to $1+q$, with summing up being $i \times (1+q)$. The expectation on $\sum_i p_i \times i$ now becomes $\sum_i p_i \times i \times (1+q)$. Therefore, after time $t+1$, the expectation of food is $(1+q)^{t+1}$.


\textbf{Switch mode.} In the current implementation, when $energyleft + foodleft x u > 2M$.
In other word, if there is enough energy for both this organism and the child it (could) reproduces.

