\section{Initial Insights and Observations}
\label{sec:initial}
The initial challenge of this class was to beat RandomPlayer, an organism which moves and reproduces randomly.  A number of incremental improvements occurred to us in the first round of brainstorming:
\begin{itemize}
\item A simple greedy strategy would consistently beat RandomPlayer.  This strategy would always move onto food in adjacent squares:
\begin{verbatim}
for (i=0 to 4) {
    if foodPresent[i] {
        move in direction i;
    } else move in random direction;
}
\end{verbatim}

\item A consistent movement strategy would always consistently beat RandomPlayer.  This strategy would consistently move in one direction, stopping to eat, and consistently reproduce in another direction.
    
\begin{verbatim}
	if (foodLeft>0) stayput;
	else if ((randInt from 0 to 10)==1) reproduce North;
	else move west;
\end{verbatim}

\item Staying still is very cheap relative to movement.  A greedy strategy like the one above the else condition resulting in stayPut rather than random movement is superior.

\item Food tends to multiply faster than it appears.  This means that there is an advantage on a single-player board to waiting for a number of rounds before beginning to move with one of the strategies above.
\end{itemize}

Essentially, we used these four insights to guide all of our future strategies:
\begin{enumerate}
\item Greedy strategies work
\item Movement should be controlled and deliberate
\item Stay still when advantage to moving is unclear
\item Allow food to accumulate to increase total board energy
\end{enumerate}
